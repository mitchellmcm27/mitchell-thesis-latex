\documentclass[11pt,fleqn,openany]{ut-thesis}
% the degree command from ut-thesis clashes with the gensymb package
\let\thesisdegree\degree  % Store \degree in \thesisdegree
\let\degree\relax

% some package redefines maketitle after this, breaking the ut-thesis command
% so we rename and and release the maketitle command
\let\makethesistitle\maketitle 
\let\maketitle\relax

\usepackage[fancy]{mitchell-thesis}

\addbibresource{somelibrary.bib}

% options for ut-thesis
\author{Mitchell McMillan}
\gradyear{2022}
\thesisdegree{Doctor of Philosophy}
\department{Earth Sciences}
\title{Construction of Orogenic Plateaus: Geomorphology and Geodynamics of the Puna Plateau, Central Andes}

\begin{document}

\frontmatter
\makethesistitle
\begin{abstract}
	\input{chapters/abstract/text.tex}
\end{abstract}

\begin{acknowledgements}
\input{frontmatter/acknowledgements.tex}
\end{acknowledgements}

\microtypesetup{protrusion=false}
\tableofcontents
\listoffigures
\listoftables
\microtypesetup{protrusion=true}

\mainmatter

\begin{refsection}
	\chapter{Introduction}\label{ch:intro}
	\input{chapters/introduction/text.tex}
	\printbibliography[heading=subbibintoc]
\end{refsection}

\begin{refsection}
	\Chapter{Large-scale Cenozoic wind erosion \\in the Puna Plateau}{The Salina del Fraile Depression\fnmark}\label{ch:wind}
	\footnotetext{%
		This chapter has been published in the \emph{Journal of Geophysical Research: Earth Surface}.\\
		Citation: \fullcite{McMillan2020}.\\
		Supplementary materials are available online (\url{https://doi.org/10.1029/2020JF005682}).}
	\input{chapters/wind/text.tex}
	\printbibliography[heading=subbibintoc]
\end{refsection}

\begin{refsection}
	\Chapter{Eocene to Quaternary deformation\\of the Southern Puna Plateau}{Thermochronology, geochronology, and  structural geology of an Andean hinterland basin (NW Argentina)\fnmark}\label{ch:thermochron}
	\footnotetext{%
		This chapter is under review in \emph{Tectonics}.\\
		Citation: \fullcite{McMillan2022c}.\\
		Supplementary material is available in \cref{ch:s3}.}
	\input{chapters/thermochron/text.tex}
	\printbibliography[heading=subbibintoc]
\end{refsection}

\begin{refsection}
	\Chapter{Two styles of lithospheric dripping}{Synthesizing gravitational instability models, \\continental tectonics, and geologic observations\fnmark}\label{ch:drips}
	\footnotetext{%
		This chapter has been prepared for submission to \emph{Geochemistry, Geophysics, Geosystems}.\\
		Citation: \fullcite{McMillan2022b}.\\
		Supplementary material is available in \cref{ch:s4}.}
	\input{chapters/drips/text.tex}
	\printbibliography[heading=subbibintoc]
\end{refsection}

\begin{refsection}
	\Chapter{Geodynamics of continental back-arcs}{Lithospheric foundering under a weak crust \\in the Central Andes\fnmark}\label{ch:model}
	\footnotetext{%
		This chapter has been prepared for submission to \emph{Geosphere}.\\
		Citation: \fullcite{McMillan2022a}.}
	\input{chapters/model/text.tex}
	\printbibliography[heading=subbibintoc]
\end{refsection}

\begin{refsection}
	\chapter{Conclusions}
	\input{chapters/conclusions/text.tex}
	\printbibliography[heading=subbibintoc]
\end{refsection}

\appendix

\chapter{Supporting information for Chapter 3}\label{ch:s3}
Tables, data, etc.
\clearpage


\chapter{Supporting information for Chapter 4}\label{ch:s4}
Tables, data, etc.
\clearpage

\backmatter

\end{document}
